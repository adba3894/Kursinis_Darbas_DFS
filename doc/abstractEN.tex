Paskirstytųjų failų sistemų tikslas yra pilnai decentralizuoti tinklą ir padaryti jį veiksminegsniu padidinant failų prieinamumą. Šios sistemos buvo sukurtos padaryti duomenų keitimąsi greitesniu ir patogesniu.

Šis darbas padės susipažinti su egzistuojančiais paskirstytųjų failų sistemų tipais ir jų architektūra. Darbas orientuotas į IPFS (Eng. InterPlanetary File System) failų sistemą, kuri bus išanalizuota detaliau. Taip pat darbe bus pateiktas parsiuntimo - išsiuntimo testų rezultatai, kurių tikslas ištestuoti sistemos našumą.